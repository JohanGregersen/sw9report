\section{INTRODUCTION}
\label{sec:intro}

The rising interest, research, and development of GPS devices have increased potential market significantly. With cheaper devices comes a larger number of potential applications. 

A constant feed of spatio-temporale points from the vehicle network, available in real-time with high accuracy, makes for an interesting market for a broader spectrum of companies.

Insurance companies are new additions to this spectrum of companies. Car insurances has traditionally been based on the facts available at the time of purchase. Insurance companies will usually require detailed information about you and your car, creating an offer based on that. Coupled with historical statistics, an insurance company can make reasonable predictions about risks associated with insuring your car. For example, statistics often show young drivers being involved in more accidents compared to other age groups\cite{accidents}.

Basing insurance pricing on such loose prediction criteria is unfair for the individual policyholder. The generalization that young people are poor drivers means that even those who drive very well, have to pay extra because of those who do not. The problem is not unrecognised, and it is also being addressed by both insurance companies and researchers (See \ref{sec:RelatedWork}).

Insurance companies have always had historical data of what makes a driver more likely to be a poor driver. Looking at this historical data will clearly state different trends\cite{url:forbes}, e.g. driving at friday or saturday nights, or being young will make you more likely to be a high-risk policyholder. Real-time data feeds offers a lot of information, such as the customers driving patterns, e.g. how steady speed the customer often has, how often he is speeding or even how hard he normally brakes.

To make insurance more fair, it makes sense to look at individuals rather than segmented groups. If risk prediction could be calculated based on the use of individual cars, that also means the policyholder could receive a fair price offer. This type of insurance is known as Usage Based Insurance (UBI) or Pay-As-You-Drive(PAYD), and a few insurance companies are already offering it as a product. By example, the American company \texttt{Progressive} sells UBI in selected states under the name \texttt{Snapshot}\cite{snapshot}. The insurance depends on their own device mounted in your cars OBD-II port, which then measures how much you drive, number of hard braking events and more.

A common factor for any UBI system is the requirement for certain hardware in the insured cars. The insurance company needs data, which the device must provide. The data is usually sent directly from the device to the insurance company, which is also the case in the before mentioned example. This model can however be a concern for the policyholder. The data can be inaccessible, since it is sent directly to the insurance company. The insurance company then decides which data the policyholder can see, and it might not be possible to verify if you are being billed correctly. Furthermore, it is not transparent which data the insurance company actually collects and stores about you and your car. In Progressive's example it can be read from their FAQ that "some devices collect location data", which may be a privacy concern, especially given that the data can not be accessed.

In this paper the authors suggest how to alleviate concerns of the policyholder, while still maintaining a logical billing scheme that works for the insurance company. The data warehouse has to been implemented with an efficient and easy querying in mind. The authors look into which data is needed by the insurance company, and how it can be sent to them without being able to track policyholders too closely. There will be a presentation of a series of metrics, made possible through collected spatio-temporal vehicle data.

The main challenges in achieving these goals lies in data management, both for supporting different metrics, but also for computing costs for billing schemes following a privacy-friendly model. It is also desirable to design a model that the user can understand and relate to, in terms of what they are being billed for.

\section{EXPERIMENTS}\label{sec:expe}
To demonstrate possible effects of the system, a policy was created using parameters and variables found reasonable by the authors. These are what the insurance company would have to decide on as well. The chosen parameters and values can be seen in tables \ref{tab:roadtypevalues}, \ref{tab:crittimevalues}, \ref{tab:speedingvalues}, \ref{tab:accelerationvalues}, \ref{tab:basevalues}, and \ref{tab:polyvalues}. 

To test this policy, and thereby the developed system, trips were handpicked from the INFATI dataset. This was done on the following criteria:

\begin{itemize}
  \item Trips should follow similar routes
  \item Trips should be similar in distance
  \item Trips should display different driving styles
\end{itemize}

From these criteria, two different test cases were picked. The trips are visualized in BLA FLERE FIGURER!!!

\begin{table}
    \begin{tabular}{ll}
    \textbf{Roadtype} & \textbf{Weight} \\ \hline
    Motorway          & 1               \\
    Trunk             & 1               \\
    Primary           & 1               \\
    Secondary         & 1.05            \\
    Tertiary          & 1.1             \\
    Unclassified      & 1.1             \\
    Residential       & 1.2             \\
    Service           & 1.2             \\ \hline
    \end{tabular}
    \caption{Roadtypes with weights}
    \label{tab:roadtypevalues}
\end{table}

\begin{table}
    \begin{tabular}{llll}
    \textbf{Active days} & \textbf{Start} & \textbf{End} & \textbf{Weight} \\ \hline
    Monday - Friday      & 07:00:00       & 09:00:00     & 1.2             \\
    Monday - Friday      & 15:00:00       & 17:00:00     & 1.15            \\
    Saturday - Sunday    & 09:00:00       & 13:00:00     & 1.025           \\
    Saturday - Sunday    & 20:00:00       & 23:59:59     & 1.15            \\
    Saturday - Sunday    & 00:00:00       & 00:04:00     & 1.4             \\ \hline
    \end{tabular}
    \caption{Critical time intervals with weights}
    \label{tab:crittimevalues}
\end{table}

\begin{table}
    \begin{tabular}{ll}
    \textbf{Interval (\%)}   & \textbf{Weight} \\ \hline
    {[}0, 10{[}        & 1.3                   \\
    {[}10, 20{[}       & 1.4                   \\
    {[}20, 30{[}       & 1.5                   \\
    {[}30, 40{[}       & 1.6                   \\
    {[}40, 50{[}       & 1.7                   \\
    {[}50, 60{[}       & 1.8                   \\
    {[}60, 70{[}       & 1.9                   \\
    {[}70, $\infty${]} & 2                     \\ \hline
    \end{tabular}
    \caption{Speeding intervals with weights}
    \label{tab:speedingvalues}
\end{table}

\begin{table}
    \begin{tabular}{ll}
    \textbf{Interval (m/s)} & \textbf{Weight} \\ \hline
    {[}0, 3{[}              & 1               \\
    {[}3, 5{[}              & 1               \\
    {[}5, 7{[}              & 1.075           \\
    {[}7, 8{[}              & 1.1             \\
    {[}8, 9{[}              & 1.2             \\
    {[}9, 10{[}             & 1.4             \\
    {[}10, 11{[}            & 1.6             \\
    {[}11, $\infty${]}      & 1.9             \\ \hline
    \end{tabular}
    \caption{Acceleration, brake and jerk intervals with weights}
    \label{tab:accelerationvalues}
\end{table}

\begin{table}
    \begin{tabular}{ll}
    \textbf{Action} & \textbf{Base weight} \\ \hline
    Acceleration    & 50                   \\
    Brake           & 75                   \\
    Jerk            & 62.5                 \\ \hline
    \end{tabular}
    \caption{Base weights for accelerations, brakes and jerks}
    \label{tab:basevalues}
\end{table}

\begin{table}
    \begin{tabular}{ll}
    \textbf{Parameter} & \textbf{Weight} \\ \hline
    A                  & 1.02            \\
    B                  & 1.05            \\
    C                  & 0               \\
    Polynomial degree  & 1.08            \\ \hline
    \end{tabular}
    \caption{Weights for all polynomial functions}
    \label{tab:polyvalues}
\end{table}
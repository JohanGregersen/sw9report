\section{TRIP SCORING}\label{sec:trip}

This section describes one of the biggest contribution of this paper, the evaluation of each trip. The final score of trip directly correlates with meters, meaning the client can get information on how many meters each of the metrics have cost him. The authors found this representation highly relatable as a representation of score.
Because of this the base score of any trip is the amount of meters driven, meaning you always as a customer pay for the length of the trip. The metrics are described in section \ref{subsec:precon}.

\subsection{Subscores} 

Subscores is a term for the scores of each individual metric. These scores are calculated individually and added to the base score. The metrics are divided into two different categories, metrics who are scored linearly and metrics who are scored polynomially. Being scored polynomially means that delinquencies are punished more and more severe as the amount increases.

Whether the score is calculated linearly or polynomial they share a calculation of a accumulated weight. Accumulated weight is a weight dependant on the distribution of delinquencies and the weight of the intervals.

$$
\sum _{ i\quad =\quad 1 }^{ n }{ { interval }_{ i }*\quad { weight }_{ i } } \eqno{(1)}
$$

$$
\sum _{ i\quad =\quad 1 }^{ n }{ { interval }_{ i }*\quad { weight }_{ i } } \eqno{(1)}
$$


\textbf{Linear Scoring} metrics are \textit{Critical Time Period} and \textit{RoadTypes}. They are considered linearly because of the fact that it is not possible for the driver to directly control the metric. The authors do not want to punish drivers for something they are not in control of. 



\textbf{Polynomial Scoring}

\subsection{Calculating Trip Scores}

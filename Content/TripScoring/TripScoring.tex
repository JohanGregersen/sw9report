\section{TRIP SCORING}\label{sec:trip}
This section describes one of the major contribution of this paper, the evaluation of how well trips are driven. The final score of a trip directly correlates with meters, meaning the customer can get information on how many meters each of the metrics have cost him. The authors found this representation highly relatable as a score representation.
Because of this, the base score of any trip is the amount of meters driven, meaning you always as a customer, pay for the length of the trip. The metrics used to extend the base score are described in Section \ref{subsec:precon}.
It is important emphasize that the entire calculation is possible, based only on an entry in the \textit{Trip Fact} table. This means the insurance companies can do the entire score calculation without any use of sensitive information.

\subsection{Subscores} 
Subscore is a term for the scores of each individual metric. These scores are calculated individually and added to the base score. The metrics are divided into two different categories. These are metrics which are scored linearly and metrics which have an additional compounding effect, scored polynomially. Being scored polynomially, means that delinquencies are punished more severely as the amount increases.

Whether the score is calculated linearly or polynomial they share a calculation of an accumulated weight. The accumulated weight is a calculation dependent on the distribution of delinquencies and the policy-defined weight of intervals.

$$
\left( \frac { \sum _{ i }^{ n }{ \left( { interval }_{ i }*\quad { weight }_{ i } \right)  }  }{ 100 }  \right) \quad -\quad 1
$$

The equation above calculates an accumulated weight by multiplying the percentage in each interval with the associated weight and dividing it with 100. This means the accumulated weight is calculated based on distribution as well. 1 is subtracted because the accumulated weight, at a later point in the calculation, is multiplied with \textit{metersdriven} as the costs should correlate with the length of the trip. For the acceleration example given in Table \ref{tab:intervalexample}, the accumulated weight would be calculated as follows;
\begin{align*}
(1.05\quad *\quad 40)\quad +\quad (1.1\quad *\quad 20)\quad +\quad \quad \\ 
  (1.15\quad *\quad 10)\quad +\quad (1.25\quad *\quad 5)\quad +\quad \quad \\
  (1.35\quad *\quad 10)\quad +\quad (1.45\quad *\quad 5)\quad +\quad \quad \\
  (1.55\quad *\quad 10)\quad \quad \quad \quad \quad \quad \quad \quad \ \ \ \quad =\ 118 \\
\end{align*}
The result of multiplying weights with the percentages in corresponding intervals is 118, which translates to every 100 accelerations counting as 118 in the system.

\begin{align*}
\frac { 118 }{ 100 } \quad -\quad 1\quad \quad \quad \quad \quad \quad \quad \quad \quad \quad \quad  =\quad 0.18
\end{align*}


This leaves us with an accumulated weight of 0.18. This equation is important to consider when dealing with both linear and polynomial scoring\\

\textbf{Linear Scoring} metrics are \textit{Critical Time Period} and \textit{RoadTypes}. They are considered linearly because of the fact that it is not possible for the driver to directly control the metric. The authors do not want to punish drivers for something they are not in control of. 
The simplicity of a linear scoring model means that the remaining equation is simply to multiply \textit{metersdriven} with the accumulated weight.

\textbf{Polynomial Scoring} metrics are \textit{MetersSped}, \textit{Acceleration}, \textit{Jerks} and \textit{Brakes}. These metrics are controllable by the user at every given point during a trip. They are graded polynomially because continually committing these delinquencies are to be frowned upon. It is difficult to drive a trip without having a couple of delinquencies, and the first delinquencies therefore have less impact than the last. As such repeat offenders are punished the most. The insurance company can define their own polynomial equation for each metric.

\begin{align*}
ax^{y} + bx + c\quad \quad \quad \quad \quad \quad \quad \quad \quad \quad \quad \\
where\quad x = AcummulatedWeight
\end{align*}

Continuing the previous example of accelerations and assuming a policy-defined polynomial equation $1.25x^{1.08} +  1.12x + 0$, the calculation would be as follows:

\begin{align*}
(1.25 * 0.18)^{1.08} +  1.12 * 0.18 + 0\quad \quad  =\ 0.40129
\end{align*}

This turned an accumulated weight of 0.18 into 0.40129. The last step in the score calculation is multiplying the new value with the amount of delinquencies, which in the running example is 80 accelerations, and a policy-defined base price for each delinquency. In the example we will use a base price for an acceleration of 50 meters.

\begin{align*}
80 * 50 * 0.40129\quad \quad \quad \quad \quad \quad \quad \quad =\ 1605.16
\end{align*}


Given 80 accelerations, the defined intervals, and insurance policy described in Table \ref{tab:intervalexample}, the cost added from accelerations would be 1605.16 meters. Note that the polynomial function is independent of the length on trip, so the insurance company might want to include \textit{metersdriven} into their polynomial setting.
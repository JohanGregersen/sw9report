\section{DATA FOUNDATION}\label{sec:datafound}

This section first and foremost describes the different sources from which the data originated, namely the INFATI dataset and OpenStreetMap. The last paragraph describes some concepts that are vital for the understanding of this paper. 

\subsection{INFATI dataset}
The INFATI dataset\cite{art:INFATI} is a collection of spatio-temporal data, collected in 2000-2001. It consist of 20 unique cars, each providing their own separated collection of data. This is a total of 1.895.085 rows of spatio-temporal car-data. The data is logged with 1Hz, and mainly involves trajectories in northern Jutland. The purpose of the INFATI\cite{art:INFATI} project was to research driver response to alert issued by a device installed in the car. A green light is shown when the car is below the speed limit. A red light is displayed when above the speed limit, additional to a womans voice saying ''you are driving too fast''.

\subsection{OpenStreetMap}
OpenStreetMap\cite{osm} is a an open source digital roadmap offered  and maintained by the OpenStreetMap community. This project makes use of the digital roadmap of Denmark, which was provided along-side the INFATI dataset\cite{art:INFATI}. The digital roadmap of Denmark contains 762.155 rows of segment information.

\subsection{Preliminary Concepts}\label{subsec:precon}
\textbf{Metrics} is a set of attributes describing the way a car travels from a position  to a destination. Measures are calculated point by point in a GPS Fact, and a point describes what has changed since the previous point. The chosen metrics are metersdriven($meters$), critical time($high risk timespans$), roadtype($high risk segments$), acceleration($m/s^2$), brake($m/s^2$), jerk($m/s^3$), meters sped($meters$), steady-speed($boolean$). Note that braking is a negative version of acceleration, it has been separated for easier understanding from a user perception.

\textbf{Policy} is an insurance scheme created by an insurance company. The cost of a given set of metrics for a trip is defined in the scheme. The cost of a trip rated by one scheme may not be the same given a different scheme. An insurance company have the option to run multiple policies, which is advantageous when insurance policies are as highly prone to change as they are. 

\textbf{delinquency} is defined as a single occurrence of a metric above a specific pre-defined threshold. The threshold is determined by the policy of the insurance company.

\textbf{Trips} are the authors term for set of continual GPS-coordinates. Whenever 3 minutes passes after the last GPS-point was logged, the trip is ended. The next GPS-point logged will be the starting point of the next trip. Trips should not be mistaken for a tripfact, which will be explained later.
\section{DATA FOUNDATION}\label{sec:datafound}

This section first and foremost describes the different data sources utilized for this project. Two sources were used, namely the INFATI dataset and OpenStreetMap. The last paragraph describes some concepts that are vital for the understanding of this paper. 

\subsection{INFATI dataset}
The INFATI dataset\cite{art:INFATI} is a collection of spatio-temporal data, collected in 2000-2001. It data is originates from 20 unique cars, each providing their own separated collection of data. The complete dataset consists of 1.895.085 rows of spatio-temporal entries. The temporal data includes both raw and map-matched coordinates, of which this project utilizes the latter. The data is high-frequency, logged with 1Hz, and mainly involves trajectories in northern Jutland. The purpose of the INFATI\cite{art:INFATI} project was to research driver response to alerts issued by a device installed in the car. A green light is shown when the car is below the speed limit. A red light is displayed when above the speed limit, additional to a womans voice saying ''you are driving too fast''.

\subsection{OpenStreetMap}
OpenStreetMap\cite{osm} is a an open source digital roadmap offered and maintained by the OpenStreetMap community. This project makes use of the digital roadmap of Denmark, which was provided along-side the INFATI dataset\cite{art:INFATI}. The digital roadmap of Denmark contains 762.155 rows of segment information. Other than the segments themselves, this project utilizes information about speed limits and roadtypes also contained in the roadmap.

\subsection{Preliminary Concepts}\label{subsec:precon}
\textbf{Metrics} are a set of attributes describing the way a car travels from a position towards a destination. The chosen metrics are meters driven, critical time periods, roadtypes, accelerations, brakes, jerks, meters sped and steady speed. It should be noted that braking is the same as negative acceleration, but has been separated for easier understanding from a user perspective. The measures required to utilize these metrics are calculated point by point in a gpsfact, which describes what has changed since the previous point (See section \ref{sec:dataware} for a description of the data warehouse).

\textbf{Trips} are the authors term for set of continual GPS-coordinates. Whenever 3 minutes passes after the last GPS-point was logged, the trip is ended. The next GPS-point logged will be the starting point of the next trip. Trips should not be mistaken for a tripfact, which will be explained later.

\textbf{Delinquency} is defined as a single occurrence of a metric above a specific pre-defined threshold. The threshold is determined by the policy of the insurance company.

\textbf{Policy} is an insurance scheme created by an insurance company. The policy describes which monetary consequences are associated with metrics. As such the scheme defines the concrete costs for registered delinquencies on a trip. The cost of a trip rated by one scheme may not be the same given a different scheme. An insurance company have the option to run multiple policies simultaneously, which is advantageous when insurance policies are as highly prone to change. 